\section{Introduction}

In this paper we try to model the propagations of waves trough an incompressible slab of material that is in an equilibrium state.
The incompressibility of the material will prevent the formation of pressure driven waves, so looking at waves in this material looks a bit like a lost job.
But through the introduction of gravity there could form gravity driven waves who can propagate through the material. 
The equation describing these waves is given by
\begin{equation}\label{eqn:diff}
\dfrac{d}{dx} \left[\rho_{0}\omega^2\dfrac{\xi}{dx}\right] -K^2 \left[\rho_{0}\omega^2 + \rho_{0}'g\right]\xi = 0.
\end{equation}
Here $ \rho_{0} $ is the density of the material, $ \omega $ the frequency' $ K $ the wave number and $ g $ the graviton.
Classically it is enough to provide one boundary condition to solve this differential equation, this will yield a solution for every value of $ \sigma $.
However we could also give a boundary condition on a latter time step, if this is the case the possible solutions to the problem reduce a lot and only typical values for $ \omega $ will be allowed.
These values for $ \omega $ are now the eigenfrequencies of this problem and will depend on the typical parameters involved in the equations.

In this paper we will solve Equation \ref{eqn:diff} numerically using python and calculate the possible eigenfrequencies corresponding to the boundary conditions $ \xi(0)=0,\xi(1)=0 $.
For the density we will only investigate a linear dependency, $ \rho_{0} = 1 + \sigma x $ .
We will not only spend some time explaining the results it self but we will cover the most important parts of the code implementation as well.