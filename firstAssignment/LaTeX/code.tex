\section{Code Implementation}


A main goal of the project was to not only create a working code but the code had to be reusable as well.
To match these design criteria i have chosen for an object orientated approach in python and made full use of the inheritance features of the language.


\subsection{Code structure}

A good program always starts out with a good structure. 
For this i decided to work with a number of python packages.
Each of these packages had a typical task and several classes in it.
In total i created 6 packages.
Lets look in more detail at the function and classes specified in each package.
\subsubsection{The Function Package}
Recall the structure of the differential equation that had to be solved \textbf{Citation needed}.
In this differential equation there where a couple of function who where specific for the differential equation.
As well as finding the function value it also had to be possible to find the derivative of some of these functions.
 
The task of the function package is to represent the functions from the differential equation.
Each of the three functions ($ \rho_0(x,\omega^2),P(x,\omega^2),Q(x,\omega^2) $) is represented by a separate class who are daughter class from the main Function class.

For example when we want for to experiment with the density of the gas, the only thing we have to do is change how the function $ \rho_0(x,\omega^2) $ is evaluated.
Based on this evaluation the derivative of the density can be calculated as well.
Standard this is done through a numerical approximation but this can be overwritten in each subclass.

\subsubsection{The Integrator Package}

This class consists of the integrators used to solve the differential equation.
In this case it is just the Runge Kutta integrator but it can be generalised to more integrators as well.
I have written the Runge Kutta integrator in the most general case.
When one specifies no arguments or insufficient arguments to the constructor  of the class it sets up a regular Runge Kutta 4th order integrator.
In contrast, when one specifies a butcher matrix the method gets adapted to the correct order.
A possible disadvantage of this approach is that it is more error prone and possibly slower as well.  

\subsubsection{The Main Package}
Needs to be worked on.

\subsubsection{The System Package}

This is a class to represent the system of differential equations. 
It contains two classes, a general system class and a daughter class to implement the given system.
The mother class is just so that the Integrator can work with it.

\subsubsection{The Test Package}

To minimise the amount of debugging time in a coding project it is essential to start early on with the testing of all the classes and methods.
This is the responsibility of this package, as it includes several classes to test separate parts of the code.
A lot of these classes are not complete but can for example give a visual test of the solution.

\subsubsection{The Worker Package}

To speed the up computations i made use of the parallel architecture of most of the modern processors and made the code easily parallelizable, this is the task of this package.
Several classes can be found in this package, one mother class en several subclasses which give an implementation of the task that has to be preformed.
The motherclass Worker provides beside some general function who have to be implemented also some aid functions.
The details of these will be explained in the next section.

\subsection{Solving the Differential Equation}

For solving a differential equation of the form
\begin{equation}
\dfrac{d }{dt} Y(x,t) = f(x,t)
\end{equation}
We need a method to integate the function $ f(x,t) $.
In this paper we implemented the Runge Kutta method of order 4 to do so.
The explicit formula for this method is given by
\begin{equation}
y_{n+1} = y_n + \frac{h}{6}(k_{1} + 2k_{2} + 2k_{3} + k_{4})
\end{equation}
where $ k_i $ are given by
\begin{align*}
k_1 &= f(t_n, y_n),\\
k_2 &= f(t_n + \tfrac{h}{2}, y_n + \tfrac{h}{2} k_1),\\
k_3 &= f(t_n + \tfrac{h}{2}, y_n + \tfrac{h}{2} k_2),\\
k_4 &= f(t_n + h, y_n + hk_3).
\end{align*}
This method is of fourth order, this means that the error of the method drops as a fourth order function.
\subsection{Finding the Eigenvalues}
Solving the differential equation is the easy part of the assignment.
Finding the eigenvalues was the real challenge and it was the task of the worker class.

Finding an eigenvalue comes basically down of finding a root of the function
\begin{equation}
F_{end}(\omega^2;\sigma,K,g),
\end{equation}
which gives the function value of the differential equation at the position 1.
I created a couple of workers each with its own way of finding this root.

\subsubsection{The Simple Stepping Worker}

The first and easiest to implement worker is the workerStupid.
A simplified version of this algorithm can be found in Algoritme \ref{alg:stupid}.
The basic idea is to start with a large enough value for $ \omega $ to start with and decrease it step by step.
If you step over a root (If the sign of the end point flips) go back and decrease the step size.
repeat this till you are close enough to the zero.
For finding the next root just step away till you are out of the tolerance boundaries and repeat the process till you find a next point using the same convention.
A huge downsides of this algorithm that it is slow and it relies on the initial conditions of the differential equations.
A main advantage of the method is that it is guaranteed to converge toward a root and it is easy to understand.

\begin{algorithm}[H]
 initialization\;
 guess = 1\;
 shrinksize = 0.9\;
 overcount = 1\;
 endP = endPoint(guess)\;
	 \While{Pend $ \leq $0}{
		guess = guess*10\;
		endP = endPoint(guess)\:
	}
	 \While{Pend $ \geq $MAXTOL}{
		previous = guess\;
		guess = previous * shrinksize\;
		endP = endPoint(guess)\;
	
	\If{abs(endP)$ \leq $MAXTOL}{
	   break\;
	   }
	 \If{endP$ \leq $0}{
	 overCount = overCount +1\;
	 shrinksize = shrinksize + 9/(10**overCount)\;
	 }
  }
 
 \caption{The simplified version of the simple stepping algorithm, the step size is far from ideal and the convergens is sometimes very slow but it is always guaranteed to converges.}
 \label{alg:stupid}
\end{algorithm}

\subsubsection{The Non Linear Worker}

An other approach i tried is to use a non linear solver already implemented in python.
The worker doing so is the workerNLS.
I tried out some predefined non linear solvers but found that the Anderson mixing algorithm worked best for this problem. \textbf{Needs citation.}
The big problem when using this solver was to guess the starting value  $ \omega $ for the iterative solver.
One of my ideas was to calculate some roots randomly, afterwards calculate the interpolated polynomial trough this points and based on this do a guess for the next root.
This however was hard to implement and did not work all of the time, so i moved away from this approach.
However this method works good if we only need to find the first root and start out by a large enough guess for $ \omega $.
The advantage of this method was it's speed.
The disadvantage was finding higher order roots, and sometimes the algorithm did not converges at all.


