\section{Results}

In this section we briefly give some results and explain our findings.
We used the standard functions for $ P,Q $ and $ \rho $.
If we used a positive sign for $ g $ no roots existed, this is the case there is a heavier fluid on top of a lighter one and was not stable.
The function calculated for the positive values for $ g $ blew up exponentially.

\subsection{Dispersion relations}

\subsubsection{Dispersion of $ K^2 $}
As a first experiment conducted we held all parameters fixed, ($ \sigma = 1, g = -1 $ ) except for the value of $ K^2 $.
For this parameters we than searched for the eigenvalues for several modes.
This yields dispersion curves of the allowed values for $ \omega $ as a function of $ K^2 $ as given in Figure \ref{fig:DispersionRelations}.
The result in this plot looks very similar to what is given in the assignment.


\subsubsection{Dispersion of $ \sigma^2 $}
Next we calculated the dispersion relation for $ \sigma^2 $. All the other parameters where again held constant $ (g = -1,K^2 = 1) $.
The result of this is given in Figure \ref{fig:DispersionRelations}.
The shape of the curves look very similar as the dispersion for $ K $, but the value for omega drops rapidly if we search for higher order solutions.


\subsubsection{Dispersion of $ g $}
The final dispersion relation we investigated was the one of $ g $ and $ \omega^2 $.
This is given in Figure \ref{fig:DispersionRelations}.
Surprisingly there looks like a linear relation between $ g $ and $ \omega^2 $, this is completely different from the previous two dispersion relations.

\begin{figure}[h!]
        \centering
        \begin{subfigure}[b]{0.49\textwidth}
                \includegraphics[width=9cm]{../src/plot/DispersionK}
\caption{The dispersion relation of $ K $ and $ \omega $. The line at the top is for the firs mode, the green line underneath is for the second mode and so on. }
                \label{fig:gull}
        \end{subfigure}%
        ~ %add desired spacing between images, e. g. ~, \quad, \qquad, \hfill etc.
          %(or a blank line to force the subfigure onto a new line)
        \begin{subfigure}[b]{0.49\textwidth}
                \includegraphics[width=9cm]{../src/plot/DispersionSigma}
\caption{The dispersion relation of $ \sigma $ and $ \omega $. The line at the top is for the firs mode, the green line underneath is for the second mode and so on.}
                \label{fig:tiger}
        \end{subfigure}\\
        
        ~ %add desired spacing between images, e. g. ~, \quad, \qquad, \hfill etc.
          %(or a blank line to force the subfigure onto a new line)
        \begin{subfigure}[b]{.5\textwidth}
                 \includegraphics[width=9cm]{../src/plot/DispersionG}
\caption{The dispersion relation of $ g $ and $ \omega $. The line at the top is for the firs mode, the green line underneath is for the second mode and so on.}
                \label{fig:mouse}
        \end{subfigure}
        \caption{Several dispersion relations for the various variables and modes of the equation.}\label{fig:DispersionRelations}
\end{figure}